\section{Maths and algorithms}
\label{sec:mathandalgo}

\lipsum[149]

\subsection{Maths}

\lipsum[143]

\subsubsection{Simple equation}

\begin{equation}\label{eq:famous-equation}
    E= M \cdot C^2
\end{equation}

\begin{MyNotationList}{with}
    \MyNotationListItem{E}{energy}{J}\\
    \MyNotationListItem{M}{mass}{kg}\\
    \MyNotationListItem{C}{speed of light}{m/s}\\
\end{MyNotationList}

The \ref{eq:famous-equation} is famous, isn't it? A good thing to do, is to describe it a little bit ! You can use MyNotationList and MyNotationListItem to give details about the variables, the units.

\subsubsection{An array of equations}

\begin{eqnarray}
   \text{The exponent is false in } & E = M \cdot C^3 
   \label{eq:wrong-exponent} \\
   \text{The exponent is also false in } & E = M \cdot C^{1.5}
   \label{eq:also-wrong-exponent}
\end{eqnarray}

The equations are labeled so you can refer to \ref{eq:wrong-exponent} and \ref{eq:also-wrong-exponent} easily. An other method is to use subequations.

\begin{subequations}
    \begin{align}
        \text{The exponent is false in } & E = M \cdot C^3  \label{eq:subequation_a} \\
        \text{The exponent is also false in } & E = M \cdot C^{1.5} \label{eq:subequation_b}
    \end{align}
\end{subequations}

In this case, both equations have the same number. You can refer to one or the other by its label. The \ref{eq:subequation_b} is false.

\subsubsection{Equations with cases}

\begin{equation}
    \text{In equation $E = M \cdot C^p$} \hspace{0.2cm} \text{if}
    \begin{cases}
        p=2  \hspace{0.2cm} \text{the equation if correct} \\
        p!=2 \hspace{0.2cm} \text{this may be wrong}\\
    \end{cases}
    \end{equation}

\subsubsection{Work with arrays and matrices}


\begin{equation}\label{eq:arrays-and-matrices}
    E
    =
    \begin{bmatrix}
        a & b & c \\
        d & e & f \\
        g & h &  i
    \end{bmatrix}
    \cdot
    \begin{bmatrix}
          dog \\
          cat \\
          bird
    \end{bmatrix}
\end{equation}

Hum, no sure about that equation \ref{eq:arrays-and-matrices}. The next equation explains why.

\begin{equation}\label{eq:arrays-and-matrices-underbrace}
    E
    =
    \underbrace{
        \begin{bmatrix}
            a & b & \cdots & c \\
            d & e & \cdots & f \\
            \vdots  & \vdots  & \ddots & \vdots  \\
            g & h & \cdots & i
        \end{bmatrix}
        \cdot
        \begin{bmatrix}
            dog \\
            cat \\
            \vdots  \\
            bird
        \end{bmatrix}
    }_\text{Makes no sense at all ! }
\end{equation}


\subsection{Algorithms}

\begin{algorithm}

    \caption{An example of an algorithm}
    \label{alg:algorithm-example}
    
    \begin{algorithmic}
        \Require You know those steps
        \Require You want to go further
        \State First step
        \State Second step
        \State Initialize $x_i$ and $y_i$
        \While {$x_i$ != 3.1415}
            \State Something
            \State Anything
            \If {$x_k$ == 3.1415}
                \State Is this pi?
                \State It could be
            \EndIf
            \ForAll{x}
                \State I found pi!
            \EndFor
        \EndWhile
        \Repeat{I say 'you' you say}
            \State Me!
        \Until{I can't anymore}
    \end{algorithmic}
\end{algorithm}