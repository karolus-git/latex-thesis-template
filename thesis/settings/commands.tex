% Shortcut to create a figure with size, filename, extension, label, position and credit. The figure is always to store in the figs folder of the current section
\newcommand{\MyFigure}[5]{
\begin{figure}[#5]
	\begin{center}
		\includegraphics[width=#1\textwidth]{figs/#2.#3}
		\caption{#4}
		\label{fig:#2}
	\end{center}
\end{figure}
}

% Renamings
\newtheorem*{remark}{My comment} % Comment
\renewcommand{\nomname}{My acronyms and my notations} % Nomenclature
\renewcommand{\listalgorithmname}{The list of algorithms} % Algorithms



% Shortcut to print the absolute value and normal value
\newcommand{\norm}[1]{\left\lVert#1\right\rVert}
\newcommand{\abs}[1]{\left\lVert#1\right\rVert}

% Construct two columns in the nomenclature
\renewcommand{\nompreamble}{\begin{multicols}{2}}
\renewcommand{\nompostamble}{\end{multicols}}

%Create a paragraph title, underlined and bold
\newcommand{\MyParagraph}[1]{\paragraph{\underline{#1}}\mbox{}\par}

% Add an acronym to the notation table (as A (for Acronym))
\newcommand{\MyAcronym}[2]{\nomenclature[A]{#1}{#2}{}}

% Add a notation to the notation table (as N (for Notation))
% If the third parameter is empty (no unit), we don't print the unit (logical...)
\newcommand{\MyNotation}[3]{
	\nomenclature[N]{$#1$}{\ifthenelse{\equal{#3}{}}{#2}{#2 en \si{#3}}}
}

% A description of the variables of an equation with units
\newenvironment{MyNotationList}[1]
  {\noindent#1  : \small \par \begin{tabular}{rll}}
  {\end{tabular}\vspace{\baselineskip}}

% An item of the NotationList and add it to the nomenclature
\newcommand{\MyNotationListItem}[3]
{
   $#1$ & #2 & in \si{#3}
   \MyNotation{$#1$}{#2}{#3}
}

% Build two groups in the nomenclature, one for acronyms, the other for notations
\renewcommand\nomgroup[1]{%
  \item[\bfseries
  \ifstrequal{#1}{N}{Notations}{%
  \ifstrequal{#1}{A}{Acronyms}}%
]}

% Add bib files
\addbibresource{biblio/bib-p1.bib}
\addbibresource{biblio/bib-p2.bib}

% Modify the font size in bib
\renewcommand*{\bibfont}{\small}

% Translations of the commands of algorithmic
% \newcommand{\algorithmicfor}{\textbf{Pour}}
% \newcommand{\algorithmicforall}{\textbf{Pour tous}}
% \newcommand{\algorithmicrepeat}{\textbf{Répéter}}
% \newcommand{\algorithmicuntil}{\textbf{Tant que}}
% \newcommand{\algorithmicwhile}{\textbf{Tant que}}
% \newcommand{\algorithmicif}{\textbf{Si}}
% \newcommand{\algorithmicthen}{\textbf{ :}}
% \newcommand{\algorithmicdo}{\textbf{ :}}
% \newcommand{\algorithmicrequire}{\textbf{Nécessite}}
% \newcommand{\algorithmictrue}{vrai}
% \newcommand{\algorithmicelse}{\textbf{Sinon}}
% \newcommand{\algorithmicelseif}{\textbf{Sinon si}}
% \newcommand{\algorithmicend}{\textbf{Fin de boucle}}
% \newcommand{\algorithmicendif}{\textbf{Fin de boucle}}
% \newcommand{\algorithmicreturn}{\textbf{Renvoyer}}